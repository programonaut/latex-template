% !TeX root = ../template.tex

% ===========
% Konfiguration der Formattierung

%Einstellen des Seitenabstand
\usepackage[left= 2.5cm,right = 2.5cm, bottom = 4 cm, top = 2.5cm]{geometry}
%zum Ändern des Zeilenabstandes (singlespacing, onehalfspacing, doublespacing), mehr auf https://www.namsu.de/Extra/pakete/Setspace.html
\usepackage[onehalfspacing]{setspace}

%Schriftart
\usepackage[T1]{fontenc}
%Schriftarten Paket in die {} einfügen
\usepackage{}


%Fußzeile
\usepackage[headsepline=1pt, footsepline=1pt]{scrlayer-scrpage}

% Kopfzeile
\renewcommand*\chapterpagestyle{scrheadings}
\pagestyle{scrheadings}
\ihead{\Vorname{} \Nachname{}}
\ohead{\Art{}}

% ===========
% Standard Pakete
\setlength{\marginparwidth}{2cm}
\usepackage{todonotes}

%Enfernt eine Fehlermeldung (KOMA-Script)
\usepackage{scrhack}

%Für das einfügen von Bildern
\usepackage{graphicx}
%Pfad zu den Bildern --> Bei Verwendung eines Bildes in includegraphics muss nur der Name des Bildes genannt werden, z.B. DHBW_Logo
\graphicspath{{resources/images/}}

% Mehrere Bilder/Tabllen in einer Figure erstellen
\usepackage{subcaption}

%Für weitere komplexere Grafiken und Positionierung, mehr auf https://www.overleaf.com/learn/latex/TikZ_package
\usepackage{tikz}

%Für schönere Tabellen, mehr auf https://www.namsu.de/Extra/pakete/Tabularx.html
\usepackage{tabularx}
%Zum einfügen von PDF Dateien, mehr auf https://www.namsu.de/Extra/pakete/Pdfpages.html
\usepackage{pdfpages}

%Für das Literaturverzeichnis, ändern von style, ändert die Art der Zitierung und des Verzeichnis, mehr auf https://www.overleaf.com/learn/latex/Biblatex_bibliography_styles
\usepackage[style = ieee]{biblatex}
\bibliography{resources/directories/bibliography}{}

%Fügt Abkürzungsverzeichnis hinzu
%Hinzufügen von "printonlyused" in eckige Klammern, um nur verwendete Abkürzungen darzustellen
\usepackage[]{acronym}

% zusätzliche Schriftzeichen der American Mathematical Society
\usepackage{amsfonts}
\usepackage{amsmath}

%Pakete für die Deutsche Sprache
\usepackage[ngerman]{babel}
\usepackage{csquotes}

%Für Code
\usepackage{float}
\usepackage{fancyvrb}

%Für Links und Hyperlinks
\usepackage[colorlinks = true,
            linkcolor = black,
            urlcolor  = black,
            citecolor = black]{hyperref}

%Eigenschaften des PDF Dokuments anpassen (Titel, Art der Arbeit, Autor)
\hypersetup{pdftitle={\Titel},
            pdfsubject={\Art},
            pdfauthor={{\Vorname} {\Nachname}}
            }


%Code Formatierung
\usepackage{listings}
\usepackage{xcolor}

\definecolor{comment}{HTML}{0A943F}
\definecolor{linenumber}{HTML}{424445}
\definecolor{keyword}{HTML}{184FDB}
\definecolor{background}{HTML}{F2F2F2}
\definecolor{string}{HTML}{DB9418}

\lstdefinestyle{code}{
    backgroundcolor=\color{background},   
    commentstyle=\color{comment},
    keywordstyle=\color{keyword},
    numberstyle=\tiny\color{linenumber},
    stringstyle=\color{string},
    basicstyle=\ttfamily\footnotesize,
    breakatwhitespace=false,         
    breaklines=true,                 
    captionpos=b,                    
    keepspaces=true,                 
    numbers=left,                    
    numbersep=5pt,                  
    showspaces=false,                
    showstringspaces=false,
    showtabs=false,                  
    tabsize=2,
    morekeywords={
    	var
    }
}
