% !TeX root = ../template.tex

\chapter{Beispiel Kapitel}
Die Datei zu diesem Kapitel findest du im Sections Ordner, als Example.tex.\\
In dieser Datei geht es darum euch einen Einblick in die Verwendung der Kommandos aus der Dokumentation Kapitel 3 zu geben \footcite{doku}.

\section{Aufzählungen}
Dort wird gezeigt wie man eine Aufzählung macht:
\begin{itemize}
 \item Erster Stichpunkt
 \item Zweiter Stichpunkt
\end{itemize}
Oder auch wie diese Nummeriert dargestellt werden können.
\begin{enumerate}
 \item Erster Stichpunkt
 \item Zweiter Stichpunkt
\end{enumerate}

\section{Bilder}
Zudem wird gezeigt wie man Bilder einfügt und diese referenzieren kann.
\begin{center}
 \begin{figure}[h!]
  \centering
  \includegraphics[scale = 0.3]{dhbw-logo}
  \caption{Dies ist ein Logo, link zu DHBW Logo ist in Pictures/downloads.txt zu finden}
  \label{fig:dhbwlogo}
 \end{figure}
\end{center}
Sowie zum Beispiel \ref{fig:dhbwlogo} referenziert wird.

\section{Zitieren}
Ein weiterer Teil ist, wie zitiert wird, ob auf \glqq{}Deutsche\grqq{} oder auf ``Englische'' Art und Weise\footcite[3.2]{doku}.

\cite[1]{doku}

\cite[1\psq]{doku}

\cite[1\psqq]{doku}

\cite[1-4]{doku}

\todo[inline]{Der Absatz-Stil sieht scheiße aus. Vorschag: parskip=full}

\section{Code}
\lstinline$var myValue$

You can write normal text and inbetween \lstinline$comes the inline listing aka code snippet$. Afterwards you can just continue writing normal text.

\begin{lstlisting}[caption={My caption}, label={lst:my-label-1}]
    import numpy as np
\end{lstlisting}

\lstinputlisting[caption={Here comes the caption},
                label={lst:my-label-2}]
                {resources/code/placeholder.py}
