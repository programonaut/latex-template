%Standard Dokumenten Einstellungen
%Einstellen der Schriftgröße
\documentclass[a4paper,12pt]{scrreport}

%Einstellen des Seitenabstand
\usepackage[left= 2.5cm,right = 2.5cm, bottom = 4 cm, top = 2.5cm]{geometry}
%zum Ändern des Zeilenabstandes (singlespacing, onehalfspacing, doublespacing), mehr auf https://www.namsu.de/Extra/pakete/Setspace.html
\usepackage[onehalfspacing]{setspace}

%Schriftart
\usepackage[T1]{fontenc}
%Schriftarten Paket in die {} einfügen
\usepackage{}

\setcounter{secnumdepth}{5}
\setcounter{tocdepth}{5}

%Fußzeile
\usepackage[headsepline=1pt, footsepline=1pt]{scrlayer-scrpage}
\renewcommand*\chapterpagestyle{scrheadings}
\pagestyle{scrheadings}
\ihead{\Vorname{} \Nachname{}}
\ohead{\Art{}}


%===============================================================================
% Standard Packages

%Für das einfügen von Bildern
\usepackage{graphicx}
%Pfad zu den Bildern --> Bei Verwendung eines Bildes in includegraphics muss nur der Name des Bildes genannt werden, z.B. DHBW_Logo
\graphicspath{{Pictures/}}
\usepackage{subcaption}

%Für weitere komplexere Grafiken und Positionierung, mehr auf https://www.overleaf.com/learn/latex/TikZ_package
\usepackage{tikz}

%Für schönere Tabellen, mehr auf https://www.namsu.de/Extra/pakete/Tabularx.html
\usepackage{tabularx}
%Zum einfügen von PDF Dateien, mehr auf https://www.namsu.de/Extra/pakete/Pdfpages.html
\usepackage{pdfpages}

%Für das Literaturverzeichnis, ändern von style, ändert die Art der Zitierung und des Verzeichnis, mehr auf https://www.overleaf.com/learn/latex/Biblatex_bibliography_styles
\usepackage[style = authortitle]{biblatex}
\bibliography{Directorys/Literatur}{}

%Fügt Abkürzungsverzeichnis hinzu
%Hinzufügen von "printonlyused" in eckige Klammern, um nur verwendete Abkürzungen darzustellen
\usepackage[]{acronym}

% zusätzliche Schriftzeichen der American Mathematical Society
\usepackage{amsfonts}
\usepackage{amsmath}

%Pakete für die Deutsche Sprache
\usepackage[ngerman]{babel}
\usepackage{csquotes}

%Für Code
\usepackage{float}
\usepackage{fancyvrb}

%Für Links und Hyperlinks
\usepackage[colorlinks = true,
            linkcolor = black,
            urlcolor  = blue,
            citecolor = black]{hyperref}

%===============================================================================
%Felder Ausfüllen, sie geben den Inhalt des Titelblatt an
%Inhalt der Titelseite
\newcommand{\Titel}{Titel der Projektarbeit}
\newcommand{\Art}{Projektarbeit}
\newcommand{\Vorname}{Max}
\newcommand{\Nachname}{Mustermann}
\newcommand{\Studiengang}{Studiengang}
\newcommand{\Abgabedatum}{01.01.1111}
\newcommand{\Bearbeitungszeitraum}{12 Wochen}
\newcommand{\Matrikelnummer}{9999999999}
\newcommand{\Kurskrzl}{Kurzkürzel}
\newcommand{\Ausbildungsfirma}{Firma}
\newcommand{\Standort}{Standort}
\newcommand{\BetreuerFirma}{Titel Vorname Nachname}
\newcommand{\BetreuerDHBW}{Titel Vorname Nachname}

%Nur benötigt wenn Sperrvermerk verwendet wird
\newcommand{\SperrvermerkAuslaufDatum}{31.12.2222}


%===============================================================================
%Start des Dokuments
\begin{document}

%Die Gliederung entspricht den Richtlinien der Informationstechnik Fakultät, bitte anpassen für die jeweiligen Richtlinien

%Titelblatt
\pagestyle{empty}
% Wichtig, damit man nicht immer von Vorlage.tex bauen muss
% !TEX root = ../Vorlage.tex

%===========Logos============================================
%DHBW_Logo
\begin{tikzpicture}[remember picture,overlay]
 \node[anchor=north east,inner xsep=50pt, inner ysep=25pt] at (current page.north east)
 {\includegraphics[scale=0.25]{DHBW_Logo}};
\end{tikzpicture}

%Firmen Logo
% \begin{tikzpicture}[remember picture,overlay]
%    \node[anchor=north west,inner xsep=50pt, inner ysep=25pt] at (current page.north west)
%               {\includegraphics[scale=0.25]{Logo1}};
% \end{tikzpicture}

%===========Inhalt===========================================
\vspace{1cm}

\begin{center}
 \large{\Titel}
\end{center}

\vspace{1cm}

\begin{center}
 \textbf{\Art}
\end{center}

\vspace{4cm}

\begin{center}
 des Studienganges \Studiengang \\
 an der Dualen Hochschule Baden-Württemberg Mannheim
\end{center}

\vspace{0.75cm}

\begin{center}
 von\\
 \Vorname{} \Nachname
\end{center}

\vspace{1cm}

\begin{center}
 \Abgabedatum
\end{center}

\vspace{2cm}

\begin{tabular}{l@{\hspace{2cm}}l}
 Bearbeitungszeitraum:            & \Bearbeitungszeitraum        \\
 Matrikelnummer, Kurs:            & \Matrikelnummer, \Kurskrzl   \\
 Ausbildungsfirma:                & \Ausbildungsfirma, \Standort \\
 Betreuer der Ausbildungsfirma:   & \BetreuerFirma               \\
 Gutachter der Dualen Hochschule: & \BetreuerDHBW                \\
\end{tabular}


%Sperrvermerk
\pagestyle{empty}
\include{Sections/02_Sperrvermerk}

\pagenumbering{Roman}

%Eigenleistung
\include{Sections/03_Eigenleistung}

%Abstract
\include{Sections/04_Abstract}

%===============================================================================
%Verzeichnisse: Verwendete Verzeichnisse Aktivieren durch entfernen von: % vor dem \
%Inhalt
\tableofcontents
\clearpage
%Abkürzung
% Wichtig, damit man nicht immer von Vorlage.tex bauen muss
% !TEX root = ../Vorlage.tex

\chapter*{Abkürzungsverzeichnis}
\begin{acronym}[slmtA]
 \acro{KDE}{K Desktop Environment}
 \acro{SQL}{Structured Query Language}
 \acro{Bash}{Bourne-again shell}
 \acro{JDK}{Java Development Kit}
 \acro{VM}{Virtuelle Maschine}
\end{acronym}

%Grafiken
%\listoffigures
%Tabellen
%\listoftables

%Vorwort: Um Vorwort zu verwenden, % entfernen
%\include{Sections/05_Vorwort}

%===============================================================================
%Weiter Gliederung:
%Einleitung (Gegenstand und Ziele der Arbeit/Aufgabenbeschreibung, Einführung in Thema, Stand der Technik/Forschung, Motivation der Aufgabenstellung/Vorausblick)
%Hauptteil (Anforderungsdefinition, Anforderungsanalyse, Lösungsgenerierung, Lösungsbewertung,Umsetzung) in sinnvollen Gliederungspunkten
%Zusammenfassung und Ausblick
\pagestyle{scrheadings}
\pagenumbering{arabic}
%===============================================================================
%Eigene Kapitel
%Kapitel als eigene .tex datei erstellen und einbinden mit: \include{Pfad/Dateiname}
\include{Sections/Example}

%===============================================================================
\pagenumbering{Roman}
%Literaturverzeichnis
\printbibliography
%Anhang

\end{document}
